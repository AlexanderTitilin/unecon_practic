\section{ВЫДЕЛЕНИЕ СООБЩЕСТВ В ГРАФЕ}
Назовем сообществом или кластером
некое разбиение множества вершин,
так же введем меру выраженности структуры сообщества,
называемую модулярностью \cite{cluster}
\begin{equation}
Q = \frac{1}{2m} \sum_{i,j} (A_{ij} - \frac{d_{i} d_{j}}{2m}) \delta(C_{i},C_{j})
\label{mod}
\end{equation} 
На рисунке \ref{mod} приведена формула модулярности,
где $m$ -- количесто ребер графа,  $A_{ij}$ элемент матрицы смежности,
$d_{i},d_{j}$ -- степени вершин графа, $\delta(C_j,C_i)$
-- функция, которая равна 0 если сообщества не равны, иначе равна 1.
\subsection{Выделение непересекающихся сообществ}
Будет рассмотрен ряд алгоритмов выделения непересекающихся сообществ,
основанный на максимализации модулярности.
\subsubsection{Label Propagation}
Данный метод основан на предположении, 
что соседние вершины находятся в одном
сообществе. Алгоритм перемещает вершины графа в то сообщество,
где находится большинство ее соседей до тех пор, когда 
вершин для перемещеня не останется.
\subsubsection{FastGreedy}
Данный алгоритм жадно максимализурет модулярность.
В начале работы алгоритма каждая вершина находится в своем сообществе.
На каждом шаге объединяются пары сообществ, чье объединение максимально
увеличивает модулярность.
\subsubsection{Сравнение алгоритмов разбиения
на неперсекающиеся сообщества}
Для сравнения эффективность алгоритмов
были созданы 10 графов согласно модели Эрдеша-Ренье на 30
вершин с параметром $p=0.5$ и 10 графов согласно модели Болобаша-Риодана
на 30 вершин. Была вычислена средняя модулярность.
\begin{table}[H]
    \caption{Результат сравнения алгоритмов}
    \label{rnd_test}
    \begin{tabular}{|c|c|c|}
        \hline
        & Label Propagation & FastGreedy\\
        \hline
        Эрдеша-Ренье & 0.188 & 0.186  \\ 
        \hline 
        Болобаша-Риодана & 0.554 & 0.642 \\ 
        \hline
    \end{tabular}
\end{table}
На рисунке \ref{rnd_test} приведены результаты сравнения.
Действительно модель Болобаша-Риодана 
создает графы, которые лучше описывают реальные сетевые графы.
\subsection{Выделение пересекающихся сообществ}
\subsubsection{DEMON}
Данный алгоритм, основан на алгоритме Label Propagation.
Назовем подграф эго-сетью вершины $v$, если
он содержист всех соседей вершины  $v$ и все ребра
смежные данным вершинам. Опишем алгоритм:
\begin{enumerate}
    \item $\mathcal{C} = \emptyset$ -- исходное разбиение
    \item для каждой вершины $v$ строим эго-сеть, 
        ее разбиваем методом Label Propagation;
    \item объединяем с сообществами из $\mathcal{C}$, 
        если отношение размера меньшего к большему сообществу меньше 
        $\epsilon$
\end{enumerate}
