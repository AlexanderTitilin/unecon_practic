\section{ГРАФЫ. ОСНОВНЫЕ ОПРЕДЕЛЕНИЯ. РЕАЛИЗАЦИЯ  АЛГОРИТМОВ. ПОДСЧЕТ ХАРАКТЕРИСТИК ГРАФА.}
\subsection{Основные определения}
Для того, чтобы реализовать граф, необходимо 
дать определение данного понятия. 
Пусть задано конечное множество вершин $V = \{v_1 \dots v_{n}\}$
и множество ребер $E = \{e_{1} \dots e_{m}\}$,
где $e_{k} = \{v_{i},v_{j}\}$. Пару $G = (V,E)$ назовем графом.
Если  $E = \{e_1^{i_1} \dots e_{m}^{i_{k}}\}$ мультимножество ребер,
то это мультиграф. Две вершины $v_1,v_2 \in V$ называются
смежными, если $\{e_1,e_2\} \in E$.
\subsection{Реализация графа.}
Рассмотрим реализацию графов с помощью языка программирования Python.
Был создан пакет <<graph\_lib>>, который содержит
реализованные классы для представления графа, визуализации и 
генерации случайных графов.

Класс <<Graph>> для реализации графа содержится в модуле <<graph.py>>.
Рассмотрим поля данного класса:
\begin{enumerate}
    \item \_adjacency\_list -- список смежности графа. Представлен с
        помощью словаря, в котором ключи -- это вершины,
        а значения -- это множества вершин, которым данная вершина смежна;
    \item \_verticies -- множество вершин графа;
    \item \_k\_edges -- количество ребер в графе;
\end{enumerate}
\begin{figure}[H] 
\begin{lstlisting}[language=Python] 
    def __init__(self):
        self._adjacency_list = defaultdict(set)
        self._vertecies = set()
        self._k_edges = 0
\end{lstlisting}  
    \caption{Инициализация графа.}
    \label{sec:initgraph}
\end{figure} 
На рисунке \ref{sec:initgraph} представлена функции 
инициализации пустого графа.
Добавление ребра определяется следущим образом:
\begin{equation}
    (V,G) \to (V \cup \{v,u\}, E \cup \{\{v,u\}\})
    \label{sec : gf_1}
\end{equation} 
Формула \ref{sec : gf_1} легко переносится
на язык программирования Python
\begin{figure}[H] 
\begin{lstlisting}[language=Python] 
    def add_vertex(self, vertex):
        self._vertecies.add(vertex)
\end{lstlisting}  
    \caption{Добавления вершины в граф.}
    \label{sec:add_1}
\end{figure} 
\begin{figure}[H] 
\begin{lstlisting}[language=Python] 
    def add_edge(self, a, b):
        self._adjacency_list[a].add(b)
        self._adjacency_list[b].add(a)
        self.add_vertex(a)
        self.add_vertex(b)
        self._k_edges += 1
\end{lstlisting}  
    \caption{Добавление ребра в граф.}
    \label{sec:add_2}
\end{figure} 
На рисунках \ref{sec:add_1}, \ref{sec:add_2}
приведены методы, которые добавляют в граф вершину и ребро.
Данные методы обобщаются на список вершин или ребер.
\subsection{Реализация алгоритмов на графах.}
Для успешного анализа графов необходимо реализовать
некоторые базовые алгоритмы. 
Рассмотрим алгоритмы обхода графа. Начнем с поиска в ширину.
\begin{figure}[H] 
\begin{lstlisting}[language=Python] 
    def bfs(self, start=1):
        if len(self._vertecies) > 0:
            visited = {start}
            queue = deque([start])
            while queue:
                a = queue.popleft()
                yield a
                for b in self._adjacency_list[a]:
                    if b not in visited:
                        visited.add(b)
                        queue.append(b)
\end{lstlisting}  
    \caption{Реализация обхода в ширина на языке программирования Python.}
    \label{sec:bfspy}
\end{figure} 
На рисунке, \ref{sec:bfspy}
приведена  реализация алгоритма обхода графа в ширину

Данный алгоритм будет использоваться для поиска кратчайших путей в графе
и поиске компонент связности.
\begin{figure}[H] 
\begin{lstlisting}[language=Python] 
    def dist(self, v, g):
        d = defaultdict(lambda: None)
        d[v] = 0
        for u1 in self.bfs(v):
            for u2 in self.neib(u1):
                if d[u2] is None:
                    d[u2] = d[u1] + 1
        return d[g]
\end{lstlisting}  
    \caption{Поиск кратчашего растояния между $v$ и  $g$}
    \label{mindist}
\end{figure} 
\begin{figure}[H] 
\begin{lstlisting}[language=Python] 
    def connected_components(self):
        visited = set()
        components = []
        for v in self.verticies():
            if v not in visited:
                component = []
                for u in self.bfs(v):
                    visited.add(u)
                    component.append(u)
                components.append(component)
        return components
\end{lstlisting}  
    \caption{Поиск компонент связности.}
    \label{concomp}
\end{figure} 
На рисунках \ref{mindist}, \ref{concomp}
представлены методы, которые используют поиск в ширину.

Так же аналогично реализован поиск в глубину и алгоритм
поиска мостов, основанный на нем.
\subsection{Вычисление характеристик графа}
Были разработаны методы 
для вычисления следущих характеристик графа:
\begin{enumerate}
    \item плотность сети;
    \item диаметр графа;
    \item cреднее кратчайшее расстояние;
    \item коэффициент кластеризации;
    \item локальный коэффициент кластеризации;
    \item средний коэффициент кластеризации;
    \item распределние степеней вершин;
    \item степень близости вершины;
\end{enumerate}
Рассмотрим каждую характеристику более подробно
\subsubsection{Плотность сети}
Плотность сети -- отношение количества ребер к
максимально возможному, $\frac{n (n-1)}{2}$, где $n$ -- количество вершин графа.
\begin{figure}[H] 
\begin{lstlisting}[language=Python] 
    def density(self):
        max_edges = self.k_vertecies() * (self.k_vertecies() - 1) // 2
        return self._k_edges / max_edges
\end{lstlisting}  
    \caption{Метод для вычисления плотности графа.}
    \label{densg}
\end{figure} 
На рисунке \ref{densg} представлен метод 
для вычисления плотности сети.
\subsubsection{Диаметр графа}
Диаметр графа -- максимальное расстояние между парами вершин.
Вычисляется следущим образом, необходимо 
пройти из каждой вершины поиском в ширину, выбрать вершину,
обход из которой пометил больше всего вершин.
\begin{figure}[H] 
\begin{lstlisting}[language=Python] 
    def diameter(self):
        diam_lst = [len(tuple(self.bfs(v))) - 1 for v in self._vertecies]
        return max(diam_lst)
\end{lstlisting}  
    \caption{Реализация вычисления диаметра графа.}
    \label{grdiam}
\end{figure} 
На рисунке \ref{grdiam} представлен метод для вычисления диаметра графа.
\subsubsection{Коэффициенты кластеризации}
Введем следущие понятия:
\begin{enumerate}
    \item треугольник -- граф состоящий из 3 вершин, степень каждой 2;
    \item вилка -- граф состоящий из 3 вершин, степень двух вершин 1,
        другой 2. Вершина степени 2 называется центром вилки;
\end{enumerate}
Рассмотрим коээициенты кластеризации:
\begin{enumerate}
    \item коэффициент кластеризации $\frac{3 \cdot \text{количество треугольников в графе}}{\text{количество вилок в графе}}$;
     \item локальный коэффициент кластеризации для
         вершины $v$ --\\  $\frac{\text{число треугольников с вершиной~} v}{\text{число вилок с центром } v}$;
    \item средний коэфициент кластеризации -- среднее арифметическое 
        локальных коэффициентов кластеризации;
\end{enumerate}
Для вычисления данных коэффициентов нужно
реализовать методы поиска числа вилок и треугольников
\begin{figure}[H] 
\begin{lstlisting}[language=Python] 
    def k_triangles(self):
        triplets = set()
        k = 0
        for v in self._vertecies:
            if self.deg(v) >= 2:
                for v1 in self.neib(v):
                    for v2 in self.neib(v):
                        triplet = frozenset((v, v1, v2))
                        if self.has_edge(v1, v2) and triplet not in triplets:
                            k += 1
                            triplets.add(triplet)
        return k
\end{lstlisting}  
    \caption{Вычисление количества треугольников в графе.}
    \label{trg1}
\end{figure} 
\begin{figure}[H] 
\begin{lstlisting}[language=Python] 
    def k_local_triangles(self, v):
        k = 0
        if self.deg(v) >= 2:
            for v1 in self.neib(v):
                for v2 in self.neib(v):
                    if v1 < v2 and self.has_edge(v1, v2):
                        k += 1
        return k
\end{lstlisting}  
    \caption{Вычисление количества треугольников,
    содержащих вершину $v$.}
    \label{trg2}
\end{figure} 
На рисунках \ref{trg1} , \ref{trg2} 
представлены методы для вычисления числа треугольников 
в графе.
\begin{figure}[H] 
\begin{lstlisting}[language=Python] 
    def k_local_forks(self, v):
        k = 0
        if self.deg(v) >= 2:
            for v1 in self.neib(v):
                for v2 in self.neib(v):
                    if v1 < v2 and not self.has_edge(v1, v2):
                        k += 1
        return k
\end{lstlisting}  
    \caption{Вычисление вилок с центром в вершине $v$}
    \label{fork_1}
\end{figure} 
\begin{figure}[H] 
\begin{lstlisting}[language=Python] 
    def k_forks(self):
        k = 0
        for v in self._vertecies:
            k += self.k_local_forks(v)
        return k
\end{lstlisting}  
    \caption{Вычисление числа вилок в графе}
    \label{fork_2}
\end{figure} 
На рисунках \ref{fork_1}, \ref{fork_2}
представлены методы для вычисления числа вилок.
\begin{figure}[H] 
\begin{lstlisting}[language=Python] 
    def cluster_k(self):
        if self.k_forks() == 0:
            return 0
        return 3 * self.k_triangles() / self.k_forks()
\end{lstlisting}  
    \caption{Вычисление коэффициента кластеризации.}
    \label{cluster_1}
\end{figure} 
\begin{figure}[H] 
\begin{lstlisting}[language=Python] 
    def local_cluster_k(self, v):
        if self.k_local_forks(v) > 0:
            return self.k_local_triangles(v) / self.k_local_forks(v)
        return 0
\end{lstlisting}  
    \caption{Вычисление локального коэффициента кластеризации.}
    \label{cluster_2}
\end{figure} 
\begin{figure}[H] 
\begin{lstlisting}[language=Python] 
    def mean_cluster_k(self):
        return sum(self.local_cluster_k(v) for v in self._vertecies) / self.k_vertecies()
\end{lstlisting}  
    \caption{Вычисление среднего коэффициента кластеризации}
    \label{cluster_3}
\end{figure} 
На рисунках \ref{cluster_1},\ref{cluster_2},\ref{cluster_3}
представлены методы для вычисления коэффициентов кластеризации.
\subsubsection{Распределение степеней.}
Необходимо для всех степеней вершин найти долю вершин,
которые имеют данную степень.
\begin{figure}[H] 
\begin{lstlisting}[language=Python] 
    def deg_distribution(self):
        result = []
        t = namedtuple("DegDistrNode", ("k", "p"))
        for d in set(self.deg_list()):
            k_d = len([1 for v in self.verticies() if self.deg(v) == d])
            result.append(t(d, k_d/self.k_vertecies()))
        return result
\end{lstlisting}  
    \caption{Вычисление распределения степеней.}
    \label{degdistr}
\end{figure} 
На рисунке \ref{degdistr}  приведен метод для вычисления
распределения степеней.
\subsubsection{Степень близости.}
Степень близости вершины $v$  --  $C(v) = \frac{n - 1}{\sum_{u} d(u,v)}$ 
где $n$ -- количество вершин в графе,  $d(u,v)$ -- 
кратчайшее расстояние от  $u$ до  $v$.
\begin{figure}[H] 
\begin{lstlisting}[language=Python] 
    def closeness(self, v):
        d = [self.dist(v, u)
             for u in self.verticies() if self.dist(v, u) is not None]
        if sum(d) == 0:
            return 0
        return (self.k_vertecies() - 1) / sum(d)
\end{lstlisting}  
    \caption{Вычисление степени близости вершины $v$.}
    \label{closeg}
\end{figure} 
На рисунке \ref{closeg} приведен метод для вычисления степени
близости вершины $v$.
\subsection{Поиск максимальной клики графа}
Кликой в графе $G = (V,E)$ называется такое подмножество  его вершин,
$C \subset V$, что $\forall u,v , u \neq v \in C : (u,v) \in E$.
Максимальной кликой называется такая клика, которая
не содержится в клике большего размера.

Задача поиска
максимальных клик, является NP-полной. Поэтому будет 
рассмотрен следущий приблизительный алгоритм поиска
максиальных клик \cite{clique}

\begin{enumerate}
    \item иницируется начальная клика, состоящая из одной вершины;
    \item если есть вершины, которые можно добавить в клику, то добавляется лучшая;
    \item если таких нет, то удаляется лучшая;
\end{enumerate}

Данный алгоритм выполняется требуемое число шагов,
максимальной кликой, является полученная клика наибольшего размера.
Определим понятие лучшей вершины. Назовем множеством
кандидатов, такое подмножество вершин графа, что
вершины не входят в клику, но имеют ребро с каждой вершиной из клики.
Лучшим кандидатом для добалвения клику является та вершина из 
множества кандидатов, которая имеет наибольшее число ребер с
другими вершинами из множетсва кандидатов. 
Лучшей вершиной для удаления является та вершина, у которой 
наименьшее число соседей из множества таких вершин,
которые не входят в клику и имееют на одно ребро меньше чем вершин в клике.

Теперь рассмотрим реализацию данного алгоритма на языке программирования
Python. Был создан класс <<MaxClique>> который при инициализации,
получается граф и начальную клику. Были разработаны следущие методы
\begin{enumerate}
    \item  <<candidates>> -- создает список кандидатов для
        добавления в текущую клику;
    \item <<best\_candidate>> -- возвращает лучшую вершину для добавления;
    \item <<add\_best>> -- добавляет в клику лучшую вершину;
    \item <<one\_missing>> -- возвращает список для определния лучшей
        вершины для удаления;
    \item <<best\_for\_delete>> -- возвращает лучшую вершину для удаления
    \item <<delete\_best>> -- удаляет лучшую вершину;
    \item <<step>> -- выполняет шаг алгоритма;
    \item <<find\_clique>> -- возвращает найденную максимальную клику;
\end{enumerate}

Рассмотрим каждый метод более подробно. 
\begin{figure}[H] 
\begin{lstlisting}[language=Python] 
    def __init__(self, g: Graph, start):
        self.graph = g
        self.clique = start
\end{lstlisting}  
    \caption{Инициализация объекта типа <<МaxClique>>.}
    \label{mxcl_1}
\end{figure} 
На рисунке \ref{mxcl_1} приведена инициализация объекта для поиска
максимальной клики.
\begin{figure}[H] 
\begin{lstlisting}[language=Python] 
    def candidates(self):
        candidates = set()
        for v in self.graph.verticies():
            f = True
            if v not in candidates:
                for u in self.clique:
                    if not self.graph.has_edge(u, v):
                        f = False
                        break
                if f:
                    candidates.add(v)
        return candidates

    def best_candidate(self):
        candidates = self.candidates()
        d = dict()
        for v in candidates:
            d[v] = sum(1 for u in candidates if self.graph.has_edge(u, v))
        return max(d)

    def add_best(self):
        self.clique.add(self.best_candidate())
\end{lstlisting}  
    \caption{Реализация методов <<candidates>>,<<best\_candidate>>,<<add\_best>>}
    \label{mxcl_2}
\end{figure} 
На рисунке \ref{mxcl_2} приведены методы для добавления лучшей вершины в клику.
\begin{figure}[H] 
\begin{lstlisting}[language=Python] 
    def one_missing(self):
        result = set()
        for v in self.graph.verticies():
            if v not in self.clique:
                k = sum(1 for u in self.clique if self.graph.has_edge(u, v))
                if len(self.clique) - 1 == k:
                    result.add(v)
        return result

    def best_for_delete(self):
        missing = self.one_missing()
        d = dict()
        for v in self.clique:
            d[v] = sum(1 for u in missing if self.graph.has_edge(v, u))
        return min(d)

    def delete_best(self):
        self.clique.remove(self.best_for_delete())
\end{lstlisting}  
    \caption{Методы <<one\_missing>>,<<best\_for\_delete>>,<<delete\_best>>}
    \label{mxcl_4}
\end{figure} 
На рисунке \ref{mxcl_4} приведены методы для удаления 
лучшей вершины из клики.
\begin{figure}[H] 
\begin{lstlisting}[language=Python] 
    def step(self):
        cand = self.candidates()
        if len(cand) != 0:
            self.add_best()
        else:
            self.delete_best()

    def find_clique(self, k_ticks):
        result = self.clique.copy()
        for _ in range(k_ticks):
            self.step()
            if len(self.clique) > len(result):
                result = self.clique.copy()
        return result
\end{lstlisting}  
    \caption{Реализация методов <<step>> и <<find\_clique>>}
    \label{mxcl_5}
\end{figure} 
На рисунке \ref{mxcl_5} приведены методы для поиска максимальной клики графа.

Данный алгоритм имеет ряд преимуществ, такие как простота и эффективность. К недостаткам можно отнести, то что находит лишь одну 
максимальную клику.
